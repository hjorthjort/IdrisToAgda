\documentclass{scrartcl}

\usepackage[utf8]{inputenc}

\usepackage{natbib}
\usepackage{hyperref}
\usepackage{csquotes}
\usepackage{listings}
\usepackage{graphicx}
\usepackage[colorinlistoftodos]{todonotes}
\usepackage{parskip}
\setlength{\parskip}{10pt}
\usepackage{tikz}
\usetikzlibrary{arrows, decorations.markings}
\usepackage{chngcntr}
\counterwithout{figure}{section}

\begin{document}

\begin{titlepage}

\centering
{\scshape\LARGE Master thesis Planning report}

% * Preliminary title.

\vspace{0.5cm}
{\huge\bfseries Source-to-source translation between Agda and Idris
  }

\vspace{2cm}
{\Large Jakob Larsson \texttt{<jakob@karljakoblarsson.com>}}

\vspace{1.0cm}
{\large Supervisor: Patrik Jansson  \\
        Examiner: Nils Anders Danielsson}

% \vspace{1.5cm}
\vspace{1.5cm}

\vfill
{\large \today}

\end{titlepage}

%%%%%%%%%%%%%%%%
% 5. Planning report
% When: 2 weeks after proposal is accepted.
% The planning report is due 2 weeks after the start of you thesis
% project (date of registration in ladok). The planning report has to
% be approved by your examiner and should be developed in close
% collaboration with your supervisor.
%
% The planning report should be a development of the thesis proposal.
% The following points are a good start:
%
% * Preliminary title.
% Source-to-source translation between Agda and Idris

%%%%%%%%%%%%%%%%

% \section{Introduction}

% * Background to the assignment. Why is it relevant?

Agda~\cite{agda} and Idris~\cite{idris} are two dependently typed programming
languages.  Agda is mainly focused on automated theorem proving while Idris
prioritizes general purpose programming.  However, while both languages have
different focus, they have most of their features in common. The type systems
are very similar.

% Mostly the same as the proposal. And read a few other dependent-types
% paper-intros, and write something similar.


% * Aim for the work. What should be accomplished?
Partial translation of Idris code to Agda code. (semantics-preserving)
Translate as much as possible of used Idris code into Agda, which type-checks,
hopefully runs and maybe have the same output.

We wish to investigate if it is possible to construct a source to source
translation of a common subset from Idris to Agda.  A further goal is to do
bidirectional translation between the languages, but this is probably too
ambitious for this project.  The first challenge is to find a common subset for
which it is possible to translate. The task is to construct a compiler which
handles as much of this subset as possible.

This would allow Idris libraries to be reused in Agda. A Idris program may be
more easily proved in Agda. With bidirectional translation it would be possible
to use Idris to interface a well-proven Agda program with the world.  It will
also increase the confidence in a given proof if it is valid in both languages.

% * Limitations. What should be left out and why?
Only Idris to Agda. Some advanced language features State the lang features
progression, and where it's resonable to land.  Code which is untranslateable
should be left as holes, with the source as comments.
Automated testing of program output is probably to hard, I need a simpler
verification criteria.

% Idris to Agda as the main step. Bidirectional is probably hard.
% Målet är att göra översättningen åt båda hållen, fast det kanske är
% rimligt att fokusera åt ett håll från början. Det är såklart väldigt
% svårt, kanske omöjligt. Men det är svårt att svara på idag hur svårt
% det är.

% * The formulation of the problem at hand and, the assignment. This should
%   include an extended version of the scientific problem definition and
%   references to knowledge within the area given in the thesis proposal.
I should mention which parts are extra hard. Like: dependent types, reccords,
classes/interfaces. implicit arguments. Maybe partial functions. But I'm not
sure which parts are yet.

I should some paper related to transpiling and cite. As well as both Idris and
Agda implementation-papers. Maybe the faq about differences between the
languages.

Dependent type-system are still very much research so they are not
very well understood. That makes this project hard since not just the output
needs to be correct, the advanced types still need to type-check. And that is
non-obvious how to do. And how do you leave useful holes where the transpiler
fail, which are possible for humans to fill in.

How to test and verify to translations? It's probably not feasible to get a lot
of code runnable without manual intervention. Can I even get it to type-check?
And what does it mean just that the code is well-typed in both languages.
Equality is hard, especially with dependent types.


% * Method of accomplishment. How should the work be carried out?
Developing from the Idris1 or Blodwen codebase, utilizing as much of the AST
and pretty printer from the Agda-source as possible.
Working step by step, first translating only very simple programs, then adding
more and more features. Representing unimplemented features with holes so that
it is always possible to translate code. It just becomes more and more complete
as I progress. It is a fun thing to count as well, the numbers of holes left
for a given program.

A small start is to just transpile small functions with non-dependent types
over Nat for example. Then we develop the implementation to support bigger
types. The goal then is to translate dependent types.  Functionality which is
not available in both languages should be represented by holes. That way the
user can implement whats missing.

If possible we wish to reuse the Idris complier front-end for our
implementation, but there is always a challenge to interface with existing
code.  Another hard task is to find the biggest common subset of the languages.
It is trivial to name a few common features, bu a useful compiler needs to
support most widely used features. But there are incompatible differences
between Agda and Idris which complicates things.

To test the implementation we will use the Sequential Decision Problem library
IdrisLibs~\footnote{https://gitlab.pik-potsdam.de/botta/IdrisLibs} which is one
of the bigger Idris codebases available. This will guide our implementation by
first supporting the features used in the core parts of IdrisLibs. And then
work to support more and more of the library.


% \section{Goals and Challenges}

% * Risk analysis and ethical considerations.
Working with existing code can be timeconsuming. Dependent types are hard.
Idris and Agda are not exactly alike, the languages and the type-systems differ
is a few subtle ways, some ways which are probably not possible to translate.
How do we measure success in a useful, efficient and possible way. Equality is
hard, especially regarding dependent types.

Ethical considerations and risks are not relevant. There will not be different
from any other programming language. Both Agda and Idris are general purpose
programming languages which can be used any program. They can also be Turing
complete if desired. If used non-turing complete there are actually less risks
since there is simpler to reason about the programs behavior. Dependet types
can be more easily formmally verified, which also means there is easier to use
formal method to find bugs to exploit, (Of course, since the bugs are the same
even if the intentions differ.) But It's easier for a user to verify untrusted
software which decreases risk.


% * Time plan.
%   The time plan should give an approximate date when the work is to be
%   finished. It should also list mandatory seminars and milestones for the
%   project with dates for critical steps that are needed to finish the work
%   (intermediate and final report, presentation, opposition etc).
When should I be finished? I need to ask Patrik what is reasonable. And what
will I do this summer?

Do I need milestones for specific parts of the "experiment", like certain
features or steps? I plan to write the report in tandem with the project, at
least work a bit on it every week, probably every other day or so.

The testing and verification is the big uncertainty right now. It should be in
the plan but I have no Idea on how to do it. And I probably won't have until
I have gotten quite far.

Do I need to give exact dates or is something like "week 5" okay?

I need to lookup mandatory seminars.
I'm signed up for the industry seminar.
I will probably need to do the writing seminars this fall. In study period 1.
I just missed the one i April.

And times for final report presentation and opposition.
Sometime in August/September hopefully.

% The time plan can be updated if and when needed, always in collaboration with
% your supervisor and upon approval of the examiner.


% \section{Approach}

% Notes
% There is a lot to be done. I first need to decide how to implement the
% transpiler. It still seems resonable to use the actual Idris implementation and
% just add a compilation target. But there are problems. I still need to figure
% out the roadmap of language features. I.e. what features to start with and what
% to delay.

% I plan to target the latest version of Agda, 2.5.4
% And I think Idris 1. The Blodwen implementation is very different so it will be
% hard to port to the new version. But today It's not ready enough yet.

% I will try to leverages as much availible code as possible for the
% implementation. Implementing my own BNC is not a good use of my time.

% The language features progression is harder to come up with. STLC, i.e. simple
% functions without dependable types is of course the first step. But from there
% on it's not clear.


% Maybe useful things to cite:
% R. Milner "Well-typed programs can't go wrong" (1978)

\bibliographystyle{plain}

\bibliography{ita}

\end{document}
