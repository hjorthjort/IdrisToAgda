\documentclass[parskip=half]{scrartcl}

\usepackage{natbib}
\usepackage{hyperref}
\usepackage{csquotes}
\usepackage{listings}
\usepackage{graphicx}
\usepackage[colorinlistoftodos]{todonotes}
% \usepackage{parskip}
% \setlength{\parskip}{10pt}
\usepackage{tikz}
\usetikzlibrary{arrows, decorations.markings}
\usepackage{chngcntr}
\counterwithout{figure}{section}
\usepackage{xcolor}
\usepackage{listings}

\lstset{language=Haskell,keywordstyle={\bfseries \color{blue}}}

\begin{document}

\begin{titlepage}

\centering
{\scshape\LARGE Master thesis Planning report}

% * Preliminary title.

\vspace{0.5cm}
{\huge\bfseries Source-to-source translation\\ from Idris to Agda
  }

\vspace{2cm}
{\Large Jakob Larsson\\}
\texttt{<jakob@karljakoblarsson.com>}

\vspace{1.0cm}
{\large Supervisor: Patrik Jansson  \\
        Examiner: Nils Anders Danielsson}

% \vspace{1.5cm}
\vspace{1.5cm}

\vfill
{\large \today}

\end{titlepage}

%%%%%%%%%%%%%%%%
% 5. Planning report
% When: 2 weeks after proposal is accepted.
% The planning report is due 2 weeks after the start of you thesis
% project (date of registration in ladok). The planning report has to
% be approved by your examiner and should be developed in close
% collaboration with your supervisor.
%
% The planning report should be a development of the thesis proposal.
% The following points are a good start:
%
% * Preliminary title.
% Source-to-source translation between Agda and Idris

%%%%%%%%%%%%%%%%

\section{Introduction}

% * Background to the assignment. Why is it relevant?
Agda~\cite{agda} and Idris~\cite{idris} are two dependently typed programming
languages.  Agda is mainly focused on automated theorem proving while Idris
prioritizes general purpose programming.  However, while both languages have
different focus, they share most of their features. Their type systems are
similar.

% Mostly the same as the proposal. And read a few other dependent types
% paper-intros, and write something similar.


% * Aim for the work. What should be accomplished?
% This part is an okay draft now.
We wish to investigate if it is possible to construct a source to source
translation of a common subset from Idris to Agda. A translation which
preserves semantics. The translation should handle as large subset of Idris as
possible, but it is unreasonable to expect to cover 100\% of the language.
A goal for further projects is to do bidirectional translation between the
languages.

This translator would allow Idris libraries to be reused in Agda. A Idris
program may be more easily proved in Agda. With bidirectional translation it
would be possible to use Idris to interface a well proven Agda program with the
world.  It will also increase the confidence in a given proof if it is valid in
both languages.  The first challenge is to find a common subset for which it is
possible to translate. The task is to construct a compiler which handles as
much of this subset as possible.


\section{Limitations}
% * Limitations. What should be left out and why?
% TODO This part needs to be improved.
Only translation from Idris to Agda will be considered in this project.
Bidirectional translation is to much work. Hopefully ideas and code can be
reused for further work on Agda to Idris translation.

% Some advanced language features. State the lang features
It is not feasible to translate every language feature and edge case.  The goal
is to cover a large subset of Idris features, but it is hard to give an exact
goal.  The translator will built step by step, supporting a bigger subset of
language features for each step. The translator should be usable early on and
then improved.
% progression, and where it's reasonable to land.

% Code which is untranslateable
There may be valid Idris code which is impossible to translate to Agda. There
may also be code where the correct translation is application dependent. Those
parts will be left as holes for the user to fill in a reasonable translation in
that specific case.
% should be left as holes, with the source as comments.

% Dependent records 

In Agda there are different levels of \texttt{Set} since the set of types can not be
part of it self. (Gödel's Incompleteness theorem) This is useful for proofs, but
will not be considered in this project.

% Automated testing of program output is probably to hard, I need a simpler
% verification criteria.
Ideally we would run both the source and translated program with randomized
input to automatically test the they return the same result. However, it is
probably unfeasible to get larger programs to type check and run without manual
intervention.  Therefore we will use weaker forms of validation.

\section{Problem}
% * The formulation of the problem at hand and, the assignment. This should
%   include an extended version of the scientific problem definition and
%   references to knowledge within the area given in the thesis proposal.

Dependently typed programming is the application of Martin-Löf
Intuitionistic Type Theory~\cite{martinlof} to practical programming.
Types are allowed to depend on values, which allows the compiler to check much
richer properties about the program than Hindley-Milner~\cite{hindley}~\cite{milner}
style types. However, how to efficiently implement a dependently typed
programming language is active research, see for
example~\cite{quantitative-type-theory}.

Agda and Idris are two of the mostly widely used dependently type programming
languages, with much in common. The Idris website states that Idris is inspired
by Agda but with more focus on practical
programming.\footnote{\url{http://docs.idris-lang.org/en/latest/faq/faq.html\#what-are-the-differences-between-agda-and-idris}}
An automatic translator form Idris to Agda source code could increase the
usefulness of both languages. The work of implementation of it can give us
greater knowledge of the differences between the type systems. As well as the
strengths and weaknesses of both languages.

The goal of this project is to construct a translator which can translate
a subset of Idris into valid Agda. The language features which are not possible
to translate will be left as holes to allow the user to provide the missing
parts.

% I should mention which parts are extra hard. Like: dependent types, records,
% classes/interfaces. Implicit arguments. Maybe partial functions. But I'm not
% sure which parts are yet.

% I should some paper related to transpiling and cite. As well as both Idris and
% Agda implementation-papers. Maybe the faq about differences between the
% languages.

Dependent type systems are still an active research topic, which makes the
translation hard.  Not just the output needs to be correct, the advanced types
still need to type check after the translation, which is hard.  Where the
translator fails we wish to leave holes for the user to provide the correct
translation. The holes must be constructed so that is possible which is
non-obvious to do.

% How to test and verify to translations? It's not feasible to get a lot
% of code runnable without manual intervention. Can I even get it to type check?
% And what does it mean just that the code is well typed in both languages.
% Equality is hard, especially with dependent types.
It is not obvious how to verify the translation, ideally the translated
programs both type check and can be run. Then it would be possible to compare
the outputs are the same. However it is hard enough to get the programs to type
check without manual intervention. The type systems are similar but not
identical. This means that even if both the source and translated program
type check we can not be sure that they represent the same semantics.
Especially if one of the programs contain holes. It is hard to reason about
equality in the context of dependent types.
% TODO Add specific Ideas about how to verify the implementation.


\section{Method of Accomplishment}
% * Method of accomplishment. How should the work be carried out?
The first step is to translate simple declarations of types and functions,
without dependent types. A subset roughly corresponding to Simply Typed Lambda
Calculus. Since dependent types are the focus of the project the next step is
to translate simple dependent types. There are several syntactic constructs
which are not obvious to translate, for example \texttt{mutual}-blocks and
\texttt{do}-notation. They are de-sugared in the Idris compiler so are not strictly
necessary to implement. However correctly translating them will increase the
readability and the usefulness of the generated code. After simple dependent
types the next step is to implement Sum and Product types.

% Developing from the Idris 1 or Blodwen codebase, utilizing as much of the AST
% and pretty printer from the Agda-source as possible.

We wish to reuse the Idris complier front-end for our implementation, by
implementing the translator as a back-end. Hopefully we can reuse the abstract
syntax tree (AST) definition and pretty-printer from the Agda implementation as
well. This means that most of the code needed is handling the actual
translation, minimizing work needed on surrounding infrastructure.  However,
there is always a challenge to interface with existing code.

% | Below is stated elsewhere.
% Another hard task is to find the biggest common subset of the languages.
% It is trivial to name a few common features, bu a useful compiler needs to
% support most widely used features. But there are incompatible differences
% between Agda and Idris which complicates things.

Working step by step, we first translate only very simple programs, then adding
more and more features. Currently unimplemented features are represented with
holes so that it is always possible to translate code. The output becomes more
and more complete as the project progresses. The number of holes left for
a given project also works as an informal method of verification, less holes
corresponds to a better translator.

To test the implementation we will use the Sequential Decision Problem library
IdrisLibs\footnote{\url{https://gitlab.pik-potsdam.de/botta/IdrisLibs}} which is one
of the bigger Idris codebases available. This will guide our implementation by
first supporting the features used in the core parts of IdrisLibs. And then
work to support more and more of the library.


\section{Challenges and Risks}
% * Risk analysis and ethical considerations.
% Project risks.
% TODO Elaborate on this. But these points are talked about earlier.
Working with existing code can be time-consuming, and pose problems not related
to the actual project.  Idris and Agda are not exactly alike, the languages and
the type systems differ is a few subtle ways, some ways which are probably not
possible to translate.  How do we measure success in a useful, efficient and
possible way. Equality is hard, especially regarding dependent types.

% Ethical risks
Ethical risks are not a big concern for this project. The wider aim is to
improve programming languages and make it easier to write correct programs.
Both Agda and Idris are general purpose programming languages which can be
used for every possible application. If desired both can be Turing-complete.

% Ethical considerations and risks are not relevant. There will not be different
% from any other programming language. Both Agda and Idris are general purpose
% programming languages which can be used any program. They can also be Turing
% complete if desired. If used non-turing complete there are actually less risks
% since there is simpler to reason about the programs behavior. Dependent types
% can be more easily formally verified, which also means there is easier to use
% formal method to find bugs to exploit, (Of course, since the bugs are the same
% even if the intentions differ.) But It's easier for a user to verify untrusted
% software which decreases risk.


\section{Time Plan}
% * Time plan.
%   The time plan should give an approximate date when the work is to be
%   finished. It should also list mandatory seminars and milestones for the
%   project with dates for critical steps that are needed to finish the work
%   (intermediate and final report, presentation, opposition etc).
% TODO Consult this with Patrik.
\textbf{TODO How precise should this be? Do I need exact dates or weeks or
a more general plan}

% When should I be finished? I need to ask Patrik what is reasonable. And what
% will I do this summer?

Do I need milestones for specific parts of the "experiment", like certain
features or steps? I plan to write the report in tandem with the project, at
least work a bit on it every week, probably every other day or so.

The testing and verification is the big uncertainty right now. It should be in
the plan but I have no Idea on how to do it. And I probably won't have until
I have gotten quite far.

% Do I need to give exact dates or is something like "week 5" okay?

I need to lookup mandatory seminars.
I'm signed up for the industry seminar.
I will probably need to do the writing seminars this fall. In study period 1.
I just missed the one i April.

And times for final report presentation and opposition.
Sometime in August/September hopefully.

\begin{itemize}
  \item Planningreport. (Tomorrow)
  \item First runnable impelmentation of STLC.
  \item First runnable Dependent types example.
  \item Decide verification method.
  \item Content complete draft of half-time report.
  \item Half-time report. (In 6 weeks?)
  \item Implementation of Sum-types.
  \item Final implementation version.
  \item Content complete draft of Final report.
  \item Presentation.
  \item Opposition.
  \item Final report.
\end{itemize}
% The time plan can be updated if and when needed, always in collaboration with
% your supervisor and upon approval of the examiner.


% \section{Approach}

% Notes
% There is a lot to be done. I first need to decide how to implement the
% transpiler. It still seems reasonable to use the actual Idris implementation and
% just add a compilation target. But there are problems. I still need to figure
% out the roadmap of language features. I.e. what features to start with and what
% to delay.

% I plan to target the latest version of Agda, 2.5.4
% And I think Idris 1. The Blodwen implementation is very different so it will be
% hard to port to the new version. But today It's not ready enough yet.

% I will try to leverages as much available code as possible for the
% implementation. Implementing my own BNC is not a good use of my time.

% The language features progression is harder to come up with. STLC, i.e. simple
% functions without dependable types is of course the first step. But from there
% on it's not clear.


% Maybe useful things to cite:
% R. Milner "Well-typed programs can't go wrong" (1978)

% ~\cite{coquand1992pattern} % Dependent pattern matching is hard
% ~\cite{{quantitative-type-theory} % Blodwen implementation

\bibliographystyle{plain}

\bibliography{ita}

\end{document}
