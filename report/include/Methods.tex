\chapter{Methods}

Idris is a large language. We need to build the transpiler in small steps. To
maximise utility of the transpiler we should prioritize the most used features,
and leave more obscure features for later.

Therefore we wrote a tool to capture usage statistics of the top level abstract
syntax tree (AST). The tool show which syntactic constructs are most often
used, and therefore should be prioritized. It also gives the user the
incentive to refactor Idris code which uses less-used language features,
which may result in cleaner and more easily understood code.

Usability oriented features are often are implemented as syntactic sugar and
not part of the core language. They are not necessary to implement in a working
transpiler, but are important for it to be a useful tool. There is a trade-off
there we need to make. If the transpiler would output code on the level of
assembly, just encoded in Agda syntax, even if correct, it would not be a useful
tool. Since we specifically aim to develop a tool whose output is easy to
understand for a human programmer, it is good to transpile some syntactic sugar.


% \subsection{Holes}

There may be valid Idris code which is not yet implemented by our transpiler.
Instead of failing, those parts will be left as holes for the user to fill in
a reasonable translation.
The holes must be constructed so that this is possible.
Consider example below, a implementation of vector concatenation, written in
Idris.
%in listing~\ref{lst:hole1}.

\begin{lstlisting}[language=Idris,label={lst:hole1},caption={}]
concat : Vec g a -> Vec g b -> Vec g (add a b)
concat Nil rest = rest
concat (Cons x rest) y = concat rest (Cons x y)
\end{lstlisting}

% TODO Patrik: be more positive here maybe.
% Somewhat done.
The first prototype of the transpiler may look like the listing below.
Simply replacing the whole program with a hole is not incorrect, but not very
useful.
% A terribly unhelpful translation is the one below. %in listing~\ref{lst:hole2}.

\begin{lstlisting}[language=Idris,label={lst:hole2},caption={}]
concat : ?
concat = ?
\end{lstlisting}

A more useful transpiler translate as big part of the programs as it can. See
this example: %in~\ref{lst:hole3}.

\begin{lstlisting}[language=Idris,label={lst:hole3},caption={}]
concat : Vec g a -> Vec g b -> Vec g (add a b)
concat Nil rest = rest
concat (Cons x rest) y = ?
\end{lstlisting}

An even better transpiler would also add more info from the compiler, for
example as a comment. It could potentially look as in this:
% listing~\ref{lst:hole4}

\begin{lstlisting}[language=Idris,label={lst:hole4},caption={}]
concat : Vec g a -> Vec g b -> Vec g (add a b)
concat Nil rest = rest
concat (Cons x rest) y = concat rest ?a
-- The type of hole `?a` is: `Vec g c`
-- Availible bindings are:
    -- add : N -> N -> N
    -- Cons : g -> Vec a n -> Vec a (Suc n)
    -- x : g
    -- y : Vec g b
    -- b : N
\end{lstlisting}


\subsection{Implementation details}

We reuse the Idris complier front-end and AST for our implementation. This
makes sure that we are able to parse all Idris code, and we don't introduce any
bugs in the parser. It could in theory save a lot of time since we don't need
to write a parser. It is however a lot of extra work to reuse real-world code.
Similarly, we reuse the AST and pretty-printer from Agda. This means that most
of the code we need to write is handling the actual translation.

% TODO Patrik:
% [if true]
% On the Agda side this means we will be able to print syntactically correct
% code and we could even use Agda-internal type checking functions to make sure
% what we have produced is type correct.  [If I remember correctly this second
% part is not yet possible due to difficulties in filling in all the data in
% the internal datatypes]
%
% It is sort of true. But not really. I think it is possible to encode
% non-valid Agda in the Agda AST. But using the compiler should be possible.
% But I have not tried it yet.

Both Agda and Idris 1 are written in Haskell.  This allows us to build both as
a single Cabal project using Stack.  This makes it easy to reuse all parts of
both languages implementations if needed.  We use Git and import both projects
as Git submodules. This makes the changes we make to the upstream
implementations explicitly recorded. It is also possible to apply our changes
on top of new versions of the upstream projects.  We take care to only make
minor changes to the upstream implementations, this should make it easier to
use newer versions in the future.  But since our code depends a lot on internal
interfaces, it will still be hard to keep up with new versions.

% TODO Is there something good in this paragraph?
% Trying to reuse the Agda AST as well has problems. The good thing is that
% Agda is designed to be able to print it's AST back out in the exact same
% representations as the input. However, this means that the parser, AST and
% pretty print has to keep track of a lot of things, not needed for the semantics
% of the program. This made it hard to reconstruct a valid AST from another
% program, we had to guess what parts of the data structure are meaningful or not
% for our use case. And construct dummy data just to keep the compiler
% happy, while not making any difference for the output. We still felt this was
% a reasonable trade off, since constructing a AST and pretty printer from
% scratch would probably be an equal amount of work, but it would be much more
% work to keep it in synch with newer versions of Agda, so the final tool would
% be less useful.


Further work on a more robust implementation could define a new intermediate
language which covers the union of Agda and Idris features.
This will make clear in the intermediate AST what features we support and which
are not implemented.
It also makes it possible to do bidirectional transpiling in the future.
We could also use QuickCheck to generate programs in this intermediate language
to test the code generation.


\subsection{Statistics tool}
Ideally we want the transpiler to handle as much real-world code as possible.
To guide our implementation we wrote a tool which parses Idris into its AST and
then records which data constructors are most often used. Then we ran this tool
on the Idris Prelude.
\footnote{ All the files in the \textit{libs/prelude/Prelude/} directory in commit \textit{61cf812}.
\url{https://github.com/idris-lang/Idris-dev/tree/61cf812e97c0cf07a9596c1d36ab5a70eb5758b2/libs/prelude/Prelude}}
See the results in table~\ref{tbl:stats}.


\begin{table}[h]
  \caption {Usage statistics of constructors in the Idris AST used in the Idris
  Prelude.}
  \label{tbl:stats}
\begin{center}
  \begin{tabular}{ l l l r }
    Variable reference       & PTerm:   &    PRef            &    38.72 \% \\
    Application              & PTerm:   &    PApp            &    25.71 \% \\
    $\rightarrow$            & PTerm:   &    PPi             &    12.05 \% \\
    Patten clauses           & PClause: &    PClauses        &    7.74  \% \\
    Type declaration         & PTerm:   &    PTy             &    5.70  \% \\
    Constant                 & PTerm:   &    PConstant       &    3.87  \% \\
    Interface implementation & PDecl:   &    PImplementation &    1.58  \% \\
    \\
                             & PTerm:   &    PRewrite        &    0.63  \% \\
                             & PDecl:   &    PDirective      &    0.53  \% \\
    Data declaration         & PTerm:   &    PPair           &    0.53  \% \\
                             & PTerm:   &    Placeholder     &    0.48  \% \\
                             & PTerm:   &    PCase           &    0.35  \% \\
    Let clause               & PTerm:   &    PLet            &    0.32  \% \\
    Data declaration         & PDecl:   &    PData           &    0.27  \% \\
    Interface declaration    & PDecl:   &    PInterface      &    0.23  \% \\
    With-clause              & PClause: &    PWith           &    0.21  \% \\
                             & PTerm:   &    PImpossible     &    0.20  \% \\
    Fixity declaration       & PDecl:   &    PFix            &    0.18  \% \\
                             & PTerm:   &    PType           &    0.18  \% \\
                             & PTerm:   &    PAlternative    &    0.17  \% \\
                             & PTerm:   &    PConstSugar     &    0.17  \% \\
    Lambda function          & PTerm:   &    PLam            &    0.13  \% \\
    Namespace declaration    & PDecl:   &    PNamespace      &    0.03  \% \\
    Record declaration       & PDecl:   &    PRecord         &    0.02  \% \\

  \end{tabular}
\end{center}
\end{table}


% We aim to have implemented ?? of the most widely used constructors when this
% project is done.


\subsection{Verification} \label{sec:methveri}

% Ideally the transpiled programs both type-check and runs, then we can compare its output to that of the source program.
% However, it is hard enough to get the programs to type check without manual intervention.
% This means that even if both the source and translated program type-check we can not be
% sure that they represent the same semantics.  Especially if one of the programs
% contain holes. It is hard to reason about equality in the context of dependent
% types.

As stated in sub-section~\ref{sec:veri}, it is hard to verify the correctness
of the transpiler, especially while it is a work in progress. It is not
feasible to create a 100\% correct transpiler in the time of this master`s
thesis, so we have to somehow verify an unfinished transpiler.

Even to verify that a single non-trivial program is transpiled correctly is
hard. Since the two languages have slightly different semantics, even a source
and transpiled program pair which seems to be the same for the human
programmer, might have different results for some edge-case. Some possible
verification methods for as single programs, in rough order of difficulty, is
enumerated below:

\begin{enumerate}
\item The program transpiles without errors.
\item Manually inspect that the type signatures are the same.
\item The transpiled program type-checks in Agda.
\item The transpiled program runs without errors.
\item The transpiled program runs with expected output.
\item The source and transpiled programs both returns the same output for
  automatically generated tests.
\item The source and transpiled programs are formally verified to have the same
  semantics.
\end{enumerate}

\todo{ TODO Beskriv att den har listan ar lite naiv. Att det finns m[nga stege
mellan varje steg. }

% \item The transpiler passes generated tests.
% \item The transpiler is formally proved to be correct.

However, we need to show that our transpiler works for different programs.
That rules out some methods which are too manual.  In a perfect world we would
have test suit with a lot of different programs translated and proven correct in
both languages. But no such test suite exists for Agda and Idris.

% redo This paragraph.
% Ideally we would run both the source and translated program with randomized
% input to automatically test the they return the same result. However, it is
% probably infeasible to get larger programs to type-check and run without manual
% intervention.  Therefore we will use weaker forms of validation.

To formally prove correctness of the translation is the ideal verification, but
it is far outside the scope of this project, so we will have to settle for less
rigorous methods. It is the safest way, and what would provide full confidence.
But hopefully we can still provide some degree of confidence in the
translation.





%%%%%%%%
% About the Types Graph
Re-using an existing major codebase is hard. Production ready code is usually
filled with special cases, performance considerations and other "works for
now"-code. The made this project harder than anticipated. While trying to
understand the Idris code base we produced a dependency graph for the types
used in the front end. This is one advantaged of strictly typed programming
that the types, their dependencies tells us a lot about the code and how it
works. And compared to write documentation, the compiler guarantees that this
information is up to date.

\newcommand{\pterm}{\textit{PTerm}}
\newcommand{\pdecel}{\textit{PDecl}}
\newcommand{\term}{\textit{Term}}
\newcommand{\tti}{\textit{TT}}

% TODO which
\todo{Insert the graph}
\todo{This is only a stream-of-conciousness draft.}
This graph of type dependencies can be found in appendix X. By looking at the
graph it is easy to see which types are central in the implementation and which
are only used peripherally. In figure~\ref{fig:types-crop} we see part of the
graph centered on \textit{Term}.

In the graph we cna clearly see that two of the most central data types are
\textit{PTerm} and \textit{Term}. \textit{PTerm} is the main data type of the
conrete syntax, i.e. what the parser produces. A program is a collection of
declarations \textit{PDecl} which are constructed out of \pterm. \pterm
contains everything which is known at parse-time. This is not enough for the
purpose of our transpiler.

% Idris internals again
The next step of compilation is elaboration and translation to \tti. \term is
the main data type of the \tti language. \tti has fully explicit types. It
contains all the information which is necessary for type-checking. A lot of
useful information is generated in the elaboration.

\textit{TT} is dependtly typed lambda-calculus augmented with algebraic data
types and pattern matching. \textit{TT} is full explicitly typed.~\cite{idris}
\textit{TT} is delibratly kept small to be sure it is correct. It is consisted
of well known parts.

The Idris compilator works in a few big passes. The big passes are themself
compromised of many smaller step. On a high-level the compilator goes through
the following stages: Parsing to PDecl-lang. Elaboration and translation to
\textit{TT}. Type checking. Further compilation to smaller languages. Then
lastly translation to the target, i.e. C or JS usually.
