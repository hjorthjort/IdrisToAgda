% CREATED BY DAVID FRISK, 2015
\chapter{Introduction}

% * Background to the assignment. Why is it relevant?
Agda~\cite{agda} and Idris~\cite{idris} are two dependently typed programming
languages.  Agda is mainly focused on interactive theorem proving while Idris
prioritizes general purpose programming.  However, while both languages have
different focus, they share most of their features. Their type systems are
similar.

% Mostly the same as the proposal. And read a few other dependent types
% paper-intros, and write something similar.


% * Aim for the work. What should be accomplished?
% This part is an okay draft now.
We wish to investigate if it is possible to construct a source to source
translation of a common subset from Idris to Agda. A translation which
preserves semantics. The translation should handle as large subset of Idris as
possible, but it is unreasonable to expect to cover 100\% of the language.
A goal for further projects is to do bidirectional translation between the
languages.

This translator would allow Idris libraries to be reused in Agda. A Idris
program may be more easily proved in Agda.  It will also increase the
confidence in a given proof if it is valid in both languages.  The first
challenge is to find a common subset for which it is possible to translate. The
task is to construct a compiler which handles as much of this subset as
possible.


\section{Limitations}
% * Limitations. What should be left out and why?
% TODO This part needs to be improved.
Only translation from Idris to Agda will be considered in this project. We will
target Idris 1.3.1 and Agda 2.6.0. It is not feasible to translate every
language feature and edge case.  The goal is to cover a large subset of Idris
features, but it is hard to give an exact goal.  The translator will built step
by step, supporting a bigger subset of language features for each step. The
translator should be usable early on and then improved.
% progression, and where it's reasonable to land.

% Code which is untranslateable
There may be valid Idris code which is impossible to translate to Agda. There
may also be code where the correct translation is application dependent. Those
parts will be left as holes for the user to fill in a reasonable translation in
that specific case.
% should be left as holes, with the source as comments.

% Dependent records

% Cummutally Set hierarchy. Type in type in Idris.
% Remove since Idris has different levels of Set as well, only it is always
% interfered and cannot be explicitly specified by the programmer.
% In Agda there are different levels of \texttt{Set} since the set of types can
% not be part of itself. This is useful for proofs, but will not be considered in
% this project.

% Automated testing of program output is probably to hard, I need a simpler
% verification criteria.

% Comment from NAD:  Man kan t ex inspektera typsignaturer manuellt och sedan
% förlita sig på att typcheckaren gör sitt jobb (om typsignaturen är
% tillräckligt precis). TODO
Här är det lite bra input. Det låter som en rimlig start.
Ideally we would run both the source and translated program with randomized
input to automatically test the they return the same result. However, it is
probably infeasible to get larger programs to type check and run without manual
intervention.  Therefore we will use weaker forms of validation.

\section{Problem}
% * The formulation of the problem at hand and, the assignment. This should
%   include an extended version of the scientific problem definition and
%   references to knowledge within the area given in the thesis proposal.

Dependently typed programming is the application of
% Martin-Löf
% TODO
% Maybe mention about other forms of type theories. Agda is explicitly based on
% Martin-Löf, it says so in the first paragraph of the documentation.
% "It is an extension of Martin-Löf’s type theory, and is the latest in the
% tradition of languages developed in the programming logic group at Chalmers."
Intuitionistic Type Theory~\cite{martinlof} to practical programming.
Types are allowed to depend on values, which allows the compiler to check much
richer properties about the program than Hindley-Milner~\cite{hindley}~\cite{milner}
style types. However, how to efficiently implement a dependently typed
programming language is active research.

Agda and Idris are two of the mostly widely used dependently type programming
languages, with much in common. The Idris website states that Idris is inspired
by Agda but with more focus on practical
programming.\footnote{\url{http://docs.idris-lang.org/en/latest/faq/faq.html\#what-are-the-differences-between-agda-and-idris}}
An automatic translator form Idris to Agda source code could increase the
usefulness of both languages. The work of implementation of it can give us
greater knowledge of the differences between the type systems. As well as the
strengths and weaknesses of both languages.

% Semi-automatic translator is maybe a more correct description.
The goal of this project is to construct a semi-automatic translator which can
translate a subset of Idris into valid Agda. The language features which are
not yet implemented will be left as holes to allow the user to provide the
missing parts.

% I should mention which parts are extra hard. Like: dependent types, records,
% classes/interfaces. Implicit arguments. Maybe partial functions. But I'm not
% sure which parts are yet.

% I should some paper related to transpiling and cite. As well as both Idris and
% Agda implementation-papers. Maybe the faq about differences between the
% languages.

Dependent type systems are still an active research topic, which makes the
translation hard.
% redo this sentence. The question is what it means and not mean if both
% programs type check.
Even if both the source and translated programs type check, it does not
necessarily mean the translation is correct.  Where the translator fails we
wish to leave holes for the user to provide the correct translation. The holes
must be constructed so that is possible which is non-obvious to do.

% How to test and verify to translations? It's not feasible to get a lot
% of code runnable without manual intervention. Can I even get it to type check?
% And what does it mean just that the code is well typed in both languages.
% Equality is hard, especially with dependent types.
It is not obvious how to verify the translation, ideally the translated
programs both type check and can be run. Then it would be possible to compare
the outputs are the same. However it is hard enough to get the programs to type
check without manual intervention. The type systems are similar but not
identical. This means that even if both the source and translated program
type check we can not be sure that they represent the same semantics.
Especially if one of the programs contain holes. It is hard to reason about
equality in the context of dependent types.
% TODO Add specific Ideas about how to verify the implementation.
% Ideas:
% - 
% - 

