\chapter{Background}


\section{Background}

% TODO Find relevant articles
A source-to-source compiler, often called a transcompiler or transpiler, is
a compiler which translates source code in one programming language into
another language, usually of similar level of abstraction. A traditional
compiler takes a higher level language as input and will output a lower level
language, often machine code.  Since Agda and Idris are similar languages we will write a transpiler.

Source-to-source transpilers exists for many language pairs. Even so, there are
no published attempts of a transpiler to and from dependently
typed languages.

% DONE Should this be here?
% No!
% Dependent type systems are still an active research topic and it is hard
% to determine if two statements in different systems are equal. This is
% a big challenge when trying to show that the transpiler works correctly.



% TODO is this a useful heading? I already talk about this two paragraphs up.
\subsection{Dependently Typed Programming}

Dependent types are types which depend on values.
This allows the compiler to check richer properties about programs than
a regular strong static type system, such as the one in Haskell.
A common example is a vector type which is indexed on the length of
the vector. This allows the programmer to encode constraints and invariants in
the type system.
This mixing of values and types goes both ways, types can appear in expression
and values can appear in type signatures.


For example see the definition of a vector (a list with a specific length) in
listing~\ref{lst:depex}. The datatype contains the length of the vector, we can
then use the length in the type signatures.
The functions \texttt{head} and \textit{tail} can only type-check if their
bodies adhere to the constraint specified in the type.
For both \texttt{head} and \textit{tail} the type-checker requires that they
are applied to non-empty vectors.


% Dependent types goes further and make it possible to specify propositions from
% predicate logic as types. In theory this allows for the compiler to formally
% verify a program.

\begin{lstlisting}[language=Haskell,label={lst:depex},caption={Definition of
a vector type in Idris, the type signatures of \textit{head} and \textit{tail}. \textit{n} is a type parameter, in this case it is a natural number.  }]
data Vector : Type -> Nat -> Type where
  Nil : {a : Type} -> Vec a 0
  Cons : {a : Type} -> {n : Nat} -> a -> Vector a n -> Vector a (n + 1)

head : Vector a (Suc n) -> a
tail : Vector a (Suc n) -> Vector a n
\end{lstlisting}


% Talk about parametrized types and indexed types, and their differences.
% It will be useful later, when talking about problems with my implementation.
% Maybe in the Idris subsection.

% \subsection{Agda and Idris}

% TODO Patrik:
% I think the head and tail part of the listing would fit better here as an
% example of the implicit argument text.  But please also include the
% corresponding Agda code (or the refactored Idris code with the arguments
% added).
%
% You may also need to explain the two aspects of hidden arguments.
% 1. Avoiding "declaration" (binding) of arguments [as in the type signatures
%    in Listing 1]
% 2. Avoiding "use" (application) of arguments [not shown, but you should
%    probably show it]

One specific difference between Idris and Agda relevant for this project is
implicit arguments. In dependently typed programming you often want to have
some type as a parameter to a function, but the compiler can often infer that
type, so it is very redundant to specify the type in every call to the function
if the compiler can infer it itself. Therefore both Agda and Idris allows for
a type argument to be specified implicit. However the syntax differ, in Agda
a implicit argument needs to be defined in the function definition. In Idris
every name in a type signature which starts with a lower case letter is assumed
to be implicit.

A feature of both languages is a so called 'hole'. This is not something unique
to dependently typed languages, but it is more useful with a more advanced
compiler. A hole is a part of program which is left unimplemented. This is
represented syntactically with a question mark in both Agda and Idris. It
is then possible to ask the compiler what the type of the expression in the
hole should be, as well as what bindings are in scope and theirs types. This
allows the user to interactively refine the program.


\subsection{Verification}\label{sec:veri}

We need to verify that the transpiler is correct. The transpiler is correct if
the transpiled program has the same semantics as the source.
This is simple to define in theory.
Let $tr$ be the transpiler.
Given two functions $s_A$ and $s_I$ which maps an Agda or Idris program
respectively, into some formal definition of semantics.
The transpiler is correct if it satisfies equation~\ref{eq:veri2}.
% TODO Patrik:
% or simply s_I = s_a . tr
% Perhaps add something like:
% Thus we could see the task of implementing a transpiler as factoring the
% semantics of (a subset of) Idris.

\begin{equation} \label{eq:veri1}
  s_A : A \rightarrow S,
  \ s_I : I \rightarrow S,
  \ tr : I \rightarrow A,
\end{equation}
\begin{equation} \label{eq:veri2}
  s_I = s_A\ .\ tr
\end{equation}

This is hard to show in practice however. Simply defining $s_A$ or $s_I$ for
a subset of the language is a big project, and way beyond the scope of this
thesis.
Further, equation~\ref{eq:veri2} needs to hold for every possible program.
It is of course impossible to enumerate every possible program in a finite
time, so we either need to construct a formal proof or test a subset of $p$.
Developing a formal correctness proof is an huge undertaking, there
exists only a handful of formally verified compilers published. Therefore we
will focus on partial verification by testing, we explore our options in
section~\ref{sec:methveri}.



\section{Previous work}
\todo{This is only a first draft}

Source level transpilation of a dependently type programming languages has not
been done before. However several translations between proofs systems have been
attempted. The benefit of reusing existing proofs in a new system is obvious.

However, in practice there are lots of implementations difficulties.

A dependently typed programming language can also be used for computer-checked
proofs thanks to the Curry-Howard correspondence. A type signature represents
a proposition and any implementation which type checks is a proof of that
proposition. Agda is designed to be both a programming language and proof
assistant. Idris is mainly designed to be a programming language. There are
several other projects which are mainly proof assistants and not useful for
general purpose programming. Some examples are: Coq, HOL and NuPRL.

There is not much work on transpiling between dependently type programming
languages, but there are a lot of work done on translating between different
proof assistants.

One interesting project is Dedukti~\cite{assaf2016dedukti}. The authors sees
the possibility to use Dedukti as a unifing framework for large parts of logic
and many proof systems. They plan to implement tranpilers/translators from most
proof systems to Dedukti. One such translator is described
in~\cite{assaf2015translating} where they translat HOL into Dedukti. They claim
to successfully translate the standard library of OpenTheory. % TODO describe OpenTheory
They also claim that their translator is more scalable
then~\cite{obua2006importing} for example.

Most previous work tries to convert the proof from one system for use in
another. \cite{felty1997hybrid} takes another apporach. The try to use HOL and
NuPRL in conjuction for one single proof.

\cite{denney2000prototype} translates from HOL to Coq using a own intermediate
representation. Specifically it tries to make the destination format readable.
This is the only other translator which has human readable destination format
as a goal besides my project.

