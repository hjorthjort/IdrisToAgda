\documentclass[parskip=half]{scrartcl}

\usepackage[style=ieee]{biblatex}
\usepackage{hyperref}
\usepackage{csquotes}
\usepackage{graphicx}
\usepackage[colorinlistoftodos]{todonotes}
% \usepackage{parskip}
% \setlength{\parskip}{10pt}
\usepackage{tikz}
\usetikzlibrary{arrows, decorations.markings}
\usepackage{chngcntr}
\counterwithout{figure}{section}
\usepackage{amsmath}
\usepackage{xcolor}
% \usepackage{listings}

\usepackage{idrislang}

\definecolor{mypink}{RGB}{228,77,77}
\lstset{language=Haskell,keywordstyle={\bfseries \color{mypink}}}
% \lstset{language=Idris,keywordstyle={\bfseries \color{mypink}}}
\lstset{literate={
  {->}{$\rightarrow$}{1}
  {=>}{$\Rightarrow$}{1}
}}
\renewcommand{\IdrisKeyword}[1]{{{\bfseries\color{mypink} #1 }}}

\addbibresource{ita.bib}

\lstdefinelanguage{Lambda}{%
  morekeywords={%
    % if,then,else,fix % keywords go here
    where, data
  },%
  morekeywords={[2]int},   % types go here
  otherkeywords={:}, % operators go here
  literate={% replace strings with symbols
    {->}{{$\rightarrow$}}{2}
    {=>}{{$\Rightarrow$}}{2}
    {lambda}{{$\lambda$}}{1}
    {forall}{{$\forall$}}{1}
  },
  basicstyle={\sffamily},
  keywordstyle={\bfseries},
  keywordstyle={[2]\itshape}, % style for types
  keepspaces,
  mathescape % optional
}[keywords,comments,strings]%

\begin{document}

\begin{titlepage}

\centering
{\scshape\LARGE Master's thesis half-time report}

% * Preliminary title.

\vspace{0.5cm}
{\huge\bfseries Source-to-source translation\\ from Idris to Agda
  }

\vspace{2cm}
{\Large Jakob Larsson\\}
\texttt{<jakob@karljakoblarsson.com>}

\vspace{1.0cm}
{\large Supervisor: Patrik Jansson  \\
        Examiner: Nils Anders Danielsson}

% \vspace{1.5cm}
\vspace{1.5cm}

\vfill
{\large \today}

\end{titlepage}

\tableofcontents
\newpage

%%%%%%%%%%%%%%%%%%%%%%%%%%%%%%%%%%%%%%%%%%%%%%%%%%%%%%%%%%%%%%%%%%%%%%%%%%%%%%%
% Plan for halftime report
% ========================

% Talk about possible verification methods. Pros and cons for each, and how
% feasible they are for use in a MSc thesis. See Wibergh's MSc thesis for
% inspo.

% Talk about differences between Agda and Idris
% Philosphically and technically. And in implementation.

% Problems with how to re-use existing code. Maybe discuss problems with using
% internal interfaces. [TODO gärna med referens till litteraturen]

% Write for someone who is taking the same masters program but may not have
% taken the same electives as you. So someone who has taken, logic in computer
% science, programming language technology and algorithms. But haven't taken AFP
% and types and programming languages. So the basics of a compiler and parser
% is known, and most algorithms used. The reader knows basic functional
% programming and some Haskell. But none of Agda or Idris

% I need to talk about the basics of dependently typed programming, for someone
% with a background in functional programming with algebraic data types (but no
% GADTs) and some logic. So I can just mention Curry-Howard in passing. The
% difference between propositional logic and predicate logic.

% But I need to talk about types parametrizied by values, and not only types.
% And how this relate to theorem proving, difference between Agda/Idris and
% theorem provers such as Coq/Isabelle, a little.  Why is dependently typed
% programming useful? And what are the problems with it?


% Tell about my progress so far:
% ------------------------------
% - Digging in Idris and Agda internals   [bugs found? "bad" design? perhaps refer to Edwin Brady and Idris 2 = Blodwen]
% - Reusing Idris parser and Agda pretty-printer
% - Converting AST-to-AST directly with `ita` [main Idris - to - Agda function]
% - Stats tool, and maybe results from running it

% Short background about Dependently typed programming and the need for it? [Yes - half a page?]
% (leads also to some of the problems: lots of type information needed for the translation
%   challenges in translation "two coupled artefacts": spec + code)

% References:
% -----------
% - Both Idris implementation papers
% - Agda presentation/implementation paper
% - Automatic Agda refactoring thesis maybe? [Definitely]
% - What more? [some SE resource about depencies, legacy code, etc.]

% [perhaps later: Consider what is long term of the learning: not just the code, but the new realisations: advice on moving forward after the project]


% From Canvas:
% ------------

% - In the halftime report you inform your examiner of the current status of your
%   thesis work so she/he can get a picture of the progress.

% - Any significant deviation from the planning report must be stated.

% - The student(s) provide the examiner with a written status report of the
%   project,e.g. a draft of the final reportthat includeswork done so far.
%   This document is usually an extension of the planning report.

% - Often, the written halftime report is combined with a short oral presentation
%   for the examiner (and the supervisor).

%%%%%%%%%%%%%%%%%%%%%%%%%%%%%%%%%%%%%%%%%%%%%%%%%%%%%%%%%%%%%%%%%%%%%%%%%%%%%%%

%%%%%%%%%%%%%%%%
% Comments from NAD. Lite feedback:
% On the planning report.

% * "Agda is mainly focused on automated theorem proving": Snarare
%   "interactive theorem proving".
% Sant, bara att ändra.
%
% * Agda 2.6.0 har nyligen släppts.
% Ja, men det var ju försjutton samma dag som jag bara fokuserade på rapporten.

% * "In Agda there are different levels of Set since the set of types can
%    not be part of itself": Idris använder något liknande:
%     http://docs.idris-lang.org/en/latest/faq/faq.html#does-idris-have-universe-polymorphism-what-is-the-type-of-type
% But it is always infered by the Idris compiler, it can never be specified by
% the user as in Agda.

%
% * "Dependently typed programming is the application of Martin-Löf
%    Intuitionistic Type Theory [3] to practical programming": Det finns
%    andra varianter av typteori.
% Ja, HTT och Cubic tex. Men kanske ändå inpirerat främst av Martin-Löf.
%
% * "However, how to efficiently implement a dependently typed programming
%    language is active research, see for example [6]": Jag undrar varför
%    du valde den referensen. Jag föreslår att du förklarar vad du menar
%    lite tydligare, eller tar bort referensen.
% Bara att ta bort den då.
%
% Är det inget mer jag behöver åtgärda än de kommentarerna är ju allt guld!

%%%%%%%%%%%%%%%%%

%      Todo List
% DONE (For now at least) Have a section about Idris and its major features in
%      the background. I want to talk about mutual blocks, infix etc later.
%      And how those features relate to Agda.

% DONE Use the standard structure in the thesis template.

% DONE Include a updated time plan in the end.

% DONE Look at the refactoring thesis again and try to emulate parts of it.

% DONE Start more general, then zoom in to my topic. YES!!! Finally done!
\section{Introduction}

% Software re-use allows programmers to not repeat themselfs which in theory
% whould increase productivity imensly. However in practice it often never pans
% out. Different technical barriers stop it. Programming language is a big such
% barrier. Libraries or tools writen in one language can not be used with others.
% It is often possible to use foreign function interface (FFI) to interop with
% external code, but it is often cumbersome and hard because different languages
% have different semantics which makes it hard when the conventions don't match.

% The paradigm is actually orthogonal to the type system.
Writing correct software is hard. Over the years several methods and languages
have been developed to make it easier. Types allow the compiler to check some
properties about a program before run-time. More advanced type systems allow
more complicated properties be checked.
One advanced types system is dependent types. It is the subject of plenty
of research.

Dependently typed programming languages, however, are far from mainstream.
There are many factors which hinders adoption. One is community fragmentation.
The dependently typed community, is split between several languages.
Idris, Agda, ATS, F* and Epigram are some of the most used.

A common obstacle for adoption of a new programming language is the
availability of quality libraries. Productive application development requires
many different libraries.  For example, network protocols, parser combinators
and cryptography.  Implementing all common libraries is a huge task.  There is
a lot of duplicated effort when every language needs to implement similar
libraries.  If it would be possible to re-use libraries between languages,
language implementers could focus on developing their language.
% instead of writing the same libraries over and over for every new language.

% Patrik: Good paragraph
Agda~\cite{agda} and Idris~\cite{idris} are two of the most used dependently
typed programming languages.
The Idris website states that Idris is inspired
by Agda but with more focus on practical
programming.\footnote{\url{http://docs.idris-lang.org/en/latest/faq/faq.html\#what-are-the-differences-between-agda-and-idris}}
Agda's main focus is on interactive theorem proving, though it is used as
a general purpose programming language as well.
However, while they have different focus, both languages share most of their features and their type systems are similar.

We wish to investigate if it is possible to construct a source-to-source
compiler from Idris to Agda. The source-to-source compiler, from here on called
transpiler, should handle a common subset of Idris and Agda.  The transpiler
would allow Idris libraries to be reused in Agda. This would increase the
number of available Agda libraries and increase the audience for every new
Idris library.

Dependent types allow for all propositions in predicate logic to be encoded as
types. A program which type-checks is a proof of the proposition encoded in
its type signature. A dependently type programming language is also a theorem
prover. If the transpiler is correct and verified it can be used to increase
the confidence of theorems. A proof which is valid in both languages is
stronger than one which is only valid in one language.
% It will increase the confidence in a proof if it is valid in both languages
% after transpiling.  It will increase the confidence of a given proof if it is
% valid in both type systems because it is unlikely that both type-checkers
% will have the same bug. This is only valid for a verified transpiler however,
% otherwise there is probably bugs in the transpiler.


% Gather everything about the goals in this section.
% \subsection{Goals}
The goal of this project is to construct a transpiler which can translate
a subset of Idris into valid Agda. The transpiler is intended as an aid to
a human programmer which adapts and Idris library for use in Agda.  It is not
intended as a automatic compiler which runs Idris programs using the Agda
runtime. Therefore we will leave untranslatable parts as holes for the user to
manually specify. This also means we aim to produce output which is easy for
the human user to understand.

% Expand
% Since dependently typed programming is often used for formal proofs it is
% important that the user can understand the program.

% The language features which are
% not yet implemented will be left as holes to allow the user to provide the
% missing parts.

% DONE Fix this paragraph
The transpiler should preserve the semantics of the translated program.
Otherwise it would not be a useful tool.  We will verify that the transpiler
preserves the program semantics.  However, this is hard to do formally,
especially with an incomplete transpiler.  The verification will be done as far
as possible.
% Even if it is possible to run both the source and the translated program and
% compare their output, is hard to prove that it is valid for all possible
% inputs.

% SOMEWHAT DONE Patrik:
% Could mention [here or later] the relation to refactoring: the translation
% can be seen as divided into two phases where the first phase is an Idris ->
% Idris refactoring [which should be possible to verify within Idris] and the
% second phase is the "refactored Idris" -> Agda part [which is harder to
% verify].

The transpiler could be seen as divided into two passes. First, a Idris to
Idris refactoring to resolve syntactic sugar and gather information from the
type-checker into the AST. The second pass is the Idris to Agda translation.
The first pass should be possible to verify within Idris. The second pass is
harder to verify.


\subsection{Scope}

It is not feasible to correctly translate between 100\% of the
languages in the scope of a master's thesis.  The languages are not the same,
so we will not work on features only present in Idris or Agda.  Only
transpiling from Idris to Agda will be considered. Bidirectional transpiling is
a interesting goal for further projects.

There will not be any consideration for run-time characteristics of the
translated program.
Thus an efficient Idris program may well be translated into a slow Agda program.

We will target Idris 1.3.1 and Agda 2.6.0.

\section{Background}

% TODO Find relevant articles
A source-to-source compiler, often called a transcompiler or transpiler, is
a compiler which translates source code in one programming language into
another language, usually of similar level of abstraction. A traditional
compiler takes a higher level language as input and will output a lower level
language, often machine code.  Since Agda and Idris are similar languages we will write a transpiler.

Source-to-source transpilers exists for many language pairs. Even so, there are
no published attempts of a transpiler to and from dependently
typed languages.

% DONE Should this be here?
% No!
% Dependent type systems are still an active research topic and it is hard
% to determine if two statements in different systems are equal. This is
% a big challenge when trying to show that the transpiler works correctly.



% TODO is this a useful heading? I already talk about this two paragraphs up.
\subsection{Dependently Typed Programming}

Dependent types are types which depend on values.
This allows the compiler to check richer properties about programs than
a regular strong static type system, such as the one in Haskell.
A common example is a vector type which is indexed on the length of
the vector. This allows the programmer to encode constraints and invariants in
the type system.
This mixing of values and types goes both ways, types can appear in expression
and values can appear in type signatures.


For example see the definition of a vector (a list with a specific length) in
listing~\ref{lst:depex}. The datatype contains the length of the vector, we can
then use the length in the type signatures.
The functions \texttt{head} and \textit{tail} can only type-check if their
bodies adhere to the constraint specified in the type.
For both \texttt{head} and \textit{tail} the type-checker requires that they
are applied to non-empty vectors.


% Dependent types goes further and make it possible to specify propositions from
% predicate logic as types. In theory this allows for the compiler to formally
% verify a program.

\begin{lstlisting}[language=Haskell,label={lst:depex},caption={Definition of
a vector type in Idris, the type signatures of \textit{head} and \textit{tail}. \textit{n} is a type parameter, in this case it is a natural number.  }]
data Vector : Type -> Nat -> Type where
  Nil : {a : Type} -> Vec a 0
  Cons : {a : Type} -> {n : Nat} -> a -> Vector a n -> Vector a (n + 1)

head : Vector a (Suc n) -> a
tail : Vector a (Suc n) -> Vector a n
\end{lstlisting}


% Talk about parametrized types and indexed types, and their differences.
% It will be useful later, when talking about problems with my implementation.
% Maybe in the Idris subsection.

% \subsection{Agda and Idris}

% TODO Patrik:
% I think the head and tail part of the listing would fit better here as an
% example of the implicit argument text.  But please also include the
% corresponding Agda code (or the refactored Idris code with the arguments
% added).
%
% You may also need to explain the two aspects of hidden arguments.
% 1. Avoiding "declaration" (binding) of arguments [as in the type signatures
%    in Listing 1]
% 2. Avoiding "use" (application) of arguments [not shown, but you should
%    probably show it]

One specific difference between Idris and Agda relevant for this project is
implicit arguments. In dependently typed programming you often want to have
some type as a parameter to a function, but the compiler can often infer that
type, so it is very redundant to specify the type in every call to the function
if the compiler can infer it itself. Therefore both Agda and Idris allows for
a type argument to be specified implicit. However the syntax differ, in Agda
a implicit argument needs to be defined in the function definition. In Idris
every name in a type signature which starts with a lower case letter is assumed
to be implicit.

A feature of both languages is a so called 'hole'. This is not something unique
to dependently typed languages, but it is more useful with a more advanced
compiler. A hole is a part of program which is left unimplemented. This is
represented syntactically with a question mark in both Agda and Idris. It
is then possible to ask the compiler what the type of the expression in the
hole should be, as well as what bindings are in scope and theirs types. This
allows the user to interactively refine the program.


\subsection{Verification}\label{sec:veri}

We need to verify that the transpiler is correct. The transpiler is correct if
the transpiled program has the same semantics as the source.
This is simple to define in theory.
Let $tr$ be the transpiler.
Given two functions $s_A$ and $s_I$ which maps an Agda or Idris program
respectively, into some formal definition of semantics.
The transpiler is correct if it satisfies equation~\ref{eq:veri2}.
% TODO Patrik:
% or simply s_I = s_a . tr
% Perhaps add something like:
% Thus we could see the task of implementing a transpiler as factoring the
% semantics of (a subset of) Idris.

\begin{equation} \label{eq:veri1}
  s_A : A \rightarrow S,
  \ s_I : I \rightarrow S,
  \ tr : I \rightarrow A,
\end{equation}
\begin{equation} \label{eq:veri2}
  s_I = s_A\ .\ tr
\end{equation}

This is hard to show in practice however. Simply defining $s_A$ or $s_I$ for
a subset of the language is a big project, and way beyond the scope of this
thesis.
Further, equation~\ref{eq:veri2} needs to hold for every possible program.
It is of course impossible to enumerate every possible program in a finite
time, so we either need to construct a formal proof or test a subset of $p$.
Developing a formal correctness proof is an huge undertaking, there
exists only a handful of formally verified compilers published. Therefore we
will focus on partial verification by testing, we explore our options in
section~\ref{sec:methveri}.


\section{Methods}
% * Method of accomplishment. How should the work be carried out?
Idris is a large language. We need to build the transpiler in small steps. To
maximise utility of the transpiler we should prioritize the most used features,
and leave more obscure features for later.

Therefore we wrote a tool to capture usage statistics of the top level abstract
syntax tree (AST). The tool show which syntactic constructs are most often
used, and therefore should be prioritized. It also gives the user the
incentive to refactor Idris code which uses less-used language features,
which may result in cleaner and more easily understood code.

Usability oriented features are often are implemented as syntactic sugar and
not part of the core language. They are not necessary to implement in a working
transpiler, but are important for it to be a useful tool. There is a trade-off
there we need to make. If the transpiler would output code on the level of
assembly, just encoded in Agda syntax, even if correct, it would not be a useful
tool. Since we specifically aim to develop a tool whose output is easy to
understand for a human programmer, it is good to transpile some syntactic sugar.


% \subsection{Holes}

There may be valid Idris code which is not yet implemented by our transpiler.
Instead of failing, those parts will be left as holes for the user to fill in
a reasonable translation.
The holes must be constructed so that this is possible.
Consider example below, a implementation of vector concatenation, written in
Idris.
%in listing~\ref{lst:hole1}.

\begin{lstlisting}[language=Idris,label={lst:hole1},caption={}]
concat : Vec g a -> Vec g b -> Vec g (add a b)
concat Nil rest = rest
concat (Cons x rest) y = concat rest (Cons x y)
\end{lstlisting}

% TODO Patrik: be more positive here maybe.
% Somewhat done.
The first prototype of the transpiler may look like the listing below.
Simply replacing the whole program with a hole is not incorrect, but not very
useful.
% A terribly unhelpful translation is the one below. %in listing~\ref{lst:hole2}.

\begin{lstlisting}[language=Idris,label={lst:hole2},caption={}]
concat : ?
concat = ?
\end{lstlisting}

A more useful transpiler translate as big part of the programs as it can. See
this example: %in~\ref{lst:hole3}.

\begin{lstlisting}[language=Idris,label={lst:hole3},caption={}]
concat : Vec g a -> Vec g b -> Vec g (add a b)
concat Nil rest = rest
concat (Cons x rest) y = ?
\end{lstlisting}

An even better transpiler would also add more info from the compiler, for
example as a comment. It could potentially look as in this:
% listing~\ref{lst:hole4}

\begin{lstlisting}[language=Idris,label={lst:hole4},caption={}]
concat : Vec g a -> Vec g b -> Vec g (add a b)
concat Nil rest = rest
concat (Cons x rest) y = concat rest ?a
-- The type of hole `?a` is: `Vec g c`
-- Availible bindings are:
    -- add : N -> N -> N
    -- Cons : g -> Vec a n -> Vec a (Suc n)
    -- x : g
    -- y : Vec g b
    -- b : N
\end{lstlisting}


\subsection{Implementation details}

We reuse the Idris complier front-end and AST for our implementation. This
makes sure that we are able to parse all Idris code, and we don't introduce any
bugs in the parser. It could in theory save a lot of time since we don't need
to write a parser. It is however a lot of extra work to reuse real-world code.
Similarly, we reuse the AST and pretty-printer from Agda. This means that most
of the code we need to write is handling the actual translation.

% TODO Patrik:
% [if true]
% On the Agda side this means we will be able to print syntactically correct
% code and we could even use Agda-internal type checking functions to make sure
% what we have produced is type correct.  [If I remember correctly this second
% part is not yet possible due to difficulties in filling in all the data in
% the internal datatypes]
%
% It is sort of true. But not really. I think it is possible to encode
% non-valid Agda in the Agda AST. But using the compiler should be possible.
% But I have not tried it yet.

Both Agda and Idris 1 are written in Haskell.  This allows us to build both as
a single Cabal project using Stack.  This makes it easy to reuse all parts of
both languages implementations if needed.  We use Git and import both projects
as Git submodules. This makes the changes we make to the upstream
implementations explicitly recorded. It is also possible to apply our changes
on top of new versions of the upstream projects.  We take care to only make
minor changes to the upstream implementations, this should make it easier to
use newer versions in the future.  But since our code depends a lot on internal
interfaces, it will still be hard to keep up with new versions.

% TODO Is there something good in this paragraph?
% Trying to reuse the Agda AST as well has problems. The good thing is that
% Agda is designed to be able to print it's AST back out in the exact same
% representations as the input. However, this means that the parser, AST and
% pretty print has to keep track of a lot of things, not needed for the semantics
% of the program. This made it hard to reconstruct a valid AST from another
% program, we had to guess what parts of the data structure are meaningful or not
% for our use case. And construct dummy data just to keep the compiler
% happy, while not making any difference for the output. We still felt this was
% a reasonable trade off, since constructing a AST and pretty printer from
% scratch would probably be an equal amount of work, but it would be much more
% work to keep it in synch with newer versions of Agda, so the final tool would
% be less useful.


Further work on a more robust implementation could define a new intermediate
language which covers the union of Agda and Idris features.
This will make clear in the intermediate AST what features we support and which
are not implemented.
It also makes it possible to do bidirectional transpiling in the future.
We could also use QuickCheck to generate programs in this intermediate language
to test the code generation.


\subsection{Statistics tool}
Ideally we want the transpiler to handle as much real-world code as possible.
To guide our implementation we wrote a tool which parses Idris into its AST and
then records which data constructors are most often used. Then we ran this tool
on the Idris Prelude.
\footnote{ All the files in the \textit{libs/prelude/Prelude/} directory in commit \textit{61cf812}.
\url{https://github.com/idris-lang/Idris-dev/tree/61cf812e97c0cf07a9596c1d36ab5a70eb5758b2/libs/prelude/Prelude}}
See the results in table~\ref{tbl:stats}.


\begin{table}[h]
  \caption {Usage statistics of constructors in the Idris AST used in the Idris
  Prelude.}
  \label{tbl:stats}
\begin{center}
  \begin{tabular}{ l l l r }
    Variable reference       & PTerm:   &    PRef            &    38.72 \% \\
    Application              & PTerm:   &    PApp            &    25.71 \% \\
    $\rightarrow$            & PTerm:   &    PPi             &    12.05 \% \\
    Patten clauses           & PClause: &    PClauses        &    7.74  \% \\
    Type declaration         & PTerm:   &    PTy             &    5.70  \% \\
    Constant                 & PTerm:   &    PConstant       &    3.87  \% \\
    Interface implementation & PDecl:   &    PImplementation &    1.58  \% \\
    \\
                             & PTerm:   &    PRewrite        &    0.63  \% \\
                             & PDecl:   &    PDirective      &    0.53  \% \\
    Data declaration         & PTerm:   &    PPair           &    0.53  \% \\
                             & PTerm:   &    Placeholder     &    0.48  \% \\
                             & PTerm:   &    PCase           &    0.35  \% \\
    Let clause               & PTerm:   &    PLet            &    0.32  \% \\
    Data declaration         & PDecl:   &    PData           &    0.27  \% \\
    Interface declaration    & PDecl:   &    PInterface      &    0.23  \% \\
    With-clause              & PClause: &    PWith           &    0.21  \% \\
                             & PTerm:   &    PImpossible     &    0.20  \% \\
    Fixity declaration       & PDecl:   &    PFix            &    0.18  \% \\
                             & PTerm:   &    PType           &    0.18  \% \\
                             & PTerm:   &    PAlternative    &    0.17  \% \\
                             & PTerm:   &    PConstSugar     &    0.17  \% \\
    Lambda function          & PTerm:   &    PLam            &    0.13  \% \\
    Namespace declaration    & PDecl:   &    PNamespace      &    0.03  \% \\
    Record declaration       & PDecl:   &    PRecord         &    0.02  \% \\

  \end{tabular}
\end{center}
\end{table}


% We aim to have implemented ?? of the most widely used constructors when this
% project is done.


\subsection{Verification} \label{sec:methveri}

% Ideally the transpiled programs both type-check and runs, then we can compare its output to that of the source program.
% However, it is hard enough to get the programs to type check without manual intervention.
% This means that even if both the source and translated program type-check we can not be
% sure that they represent the same semantics.  Especially if one of the programs
% contain holes. It is hard to reason about equality in the context of dependent
% types.

As stated in sub-section~\ref{sec:veri}, it is hard to verify the correctness
of the transpiler, especially while it is a work in progress. It is not
feasible to create a 100\% correct transpiler in the time of this master`s
thesis, so we have to somehow verify an unfinished transpiler.

Even to verify that a single non-trivial program is transpiled correctly is
hard. Since the two languages have slightly different semantics, even a source
and transpiled program pair which seems to be the same for the human
programmer, might have different results for some edge-case. Some possible
verification methods for as single programs, in rough order of difficulty, is
enumerated below:

\begin{enumerate}
\item The program transpiles without errors.
\item The transpiled program type-checks in Agda.
\item Manually inspect that the type signatures are the same.
\item The transpiled program runs without errors.
\item The transpiled program runs with expected output.
\item The source and transpiled programs both returns the same output for
  automatically generated tests.
\item The source and transpiled programs are formally verified to have the same
  semantics.
\end{enumerate}

% \item The transpiler passes generated tests.
% \item The transpiler is formally proved to be correct.

However, we need to show that our transpiler works for different programs.
That rules out some methods which are too manual.  In a perfect world we would
have test suit with a lot of different programs translated and proven correct in
both languages. But no such test suite exists for Agda and Idris.

% redo This paragraph.
% Ideally we would run both the source and translated program with randomized
% input to automatically test the they return the same result. However, it is
% probably infeasible to get larger programs to type-check and run without manual
% intervention.  Therefore we will use weaker forms of validation.

To formally prove correctness of the translation is the ideal verification, but
it is far outside the scope of this project, so we will have to settle for less
rigorous methods. It is the safest way, and what would provide full confidence.
But hopefully we can still provide some degree of confidence in the
translation.


\section{Results}

% Somewhat done. Patrik:
% This is relevant text, but should perhaps be split into proper results, and
% comments / discussion.  Concretely, what is before "This is good" are clearly
% results, but then you switch to "discussion" in the middle of a paragraph.
% Before that I would have expected some measure of how much of the SDP
% implementation in IdrisLibs (if any) can be translated at this stage [the
% planning report explicitly mentioned this - the (half-time) report should
% therefore relate to that].
%
% This is somewhat done.

We have constructed a transpiler from a subset of Idris to Agda. It handles the
expression language, type signatures and data declarations.  It fails for
implicit arguments, which have to be declared in Agda but not in Idris. The
transpiler re-uses both the Idris parser and the Agda pretty-printer from their
main implementations.

% TODO Maybe split into proper results/discussion.
Currently it is not possible to run the transpiler on most parts of IdrisLibs.
This is due to some bugs with relative imports when loading modules. The
transpiler is not searching for modules in the same places as the main Idris
compiler.

We have made some progress, but not as much as planned. It has taken quite some
time working with and around the existing code.  Real-life implementations make
a lot of practical considerations which complicates the code a lot, and are not
relevant for this project.

% We have developed a tool to translate simple Idris programs into Agda. It works
% for a subset of Idris, namely:

The problem of building both project with the same version of GHC and
dependencies took time. It was a manual process of comparing cabal-files and
trying to find versions of dependencies which matches both. And in some cases
change one project to use a more recent version of a dependency with a updated
interface. Idris also uses a custom build process which need some revision to
work in a new folder structure.

% Idris uses a custom build process which was a pain to use with a different
% folder structure. Since we wanted to pull in both Idris and Agda as `git`
% submodules in our repo we needed to change the folder structure and the build
% process. The was maybe not worth the effort.


We have not yet done any verification beyond manual inspection.  But transpiled
programs type-checks in Idris. It is one verification step.  But the goal is
the have a more thorough verification in the end of the project.

The development process in the beginning the process was to run the transpiler , then try to load the program in Agda and see if it compiles.
Then we would manually fix the program until it compiles.
After that we changed the transpiler to perform the same change we just did
manually.

% Implicit arguments are only known and calculated in the elaboration step of
% Idris compilation. There for it is hard to translate them after only parsing.

% TODO Write this better
% Somewhat done Patrik:
% Some overlap
The transpiler does not handle implicit arguments yet.
% Agda requires more explicit definitions, implicit arguments have to be defined.
% Idris automatically considers all lower case variables in type signatures to be
% a implicit variable.
Since the implicit arguments in Idris are elaborated in the Type-checker we can
not use the parser output to reconstruct them.
Only after Idris is compiled to a lower level intermediate
language called \texttt{tt\_elab} the type-checking and elaboration is run.

Therefore we need to run the elaboration and then extract the implicit
arguments from the intermediate representation before translating them to
Agda.  We are going to do this in a pre-precessing step before the translation,
we translate the intermediate back into the high level AST Idris. This makes it
possible the then just use the main transpiler.  This has the side effect of
making that translation useful for a Idris automatic refactoring tool. It is
often useful to change between implicit and explicit arguments when developing
a dependently typed program.

% This was used to guide and prioritize the implementation of the main tool. For
% the most part it matches our intuitive guess. But it gives us a better argument
% for that the transpiler is useful, even though it is unfinished.

% \subsection{Implementation difficulties}
% As expected we ran it to a lot of non project related difficulties when trying
% to reuse existing Agda and Idris sources.

% Should this be in the report? No!
% The Idris compiler implementation uses the state monad a lot.
% It is almost like an imperative program, just written in Haskell. The
% often touted benefits of functional programming goes out of the window, but the
% compiler is still happy. It goes to show that functional programming is not
% a miracle cure for bad programs, that is still up to the programmer.

% Something something about the crazy Idris implementation with a big state
% monad used everywhere. It is almost an imperative program,
% just written in Haskell.

% The next version of Idris is developed from scratch in a new project, and some
% of the reasons for that is the current implantation, and the difficulties in
% working with it.

% Just trying to find the internal interface of the Idris parser took a long
% time.

\section{Conclusions}

No conclusions in the halftime report.
% Nothing right now. Maybe leave this section out for the halftime report.

\newpage
\section{Updated Time Plan}

\subsubsection*{Technical work}
\begin{itemize}

  \item \textit{\color{gray}}[Past] First runnable Dependent types example.
  \item \textit{\color{gray}}[Past] Decide verification method.
  \item (First week in September) Implementation of Implicit arguments done.
  \item (Third week of September) Final implementation version.
  \item (Last week of September) Final verification of the transpiler.
\end{itemize}

\subsubsection*{Writing}
\begin{itemize}
  \item \textit{\color{gray}}[Past] Planning report.
  \item \textit{\color{gray}}[Past] Content complete draft of half-time report.
  \item \textit{\color{gray}}[Past] Final draft of half-time report to supervisor.
  \item (2019-08-10) Half-time report done.
  \item (First week in September) Content complete draft of Background in Final
    report.
  \item (Second week in September) Content complete draft of Final report.
  \item (Fourth week in September) Complete draft of Final report to supervisor
  \item (Second week in October) Final report.
\end{itemize}

\subsubsection*{Compulsory events}
\begin{itemize}
  \item \textit{\color{gray}}[Past] Industry and career advancement seminar
  \item (September) Writing seminar I
  \item (September) Writing seminar II
  \item (September) Opposition.
  \item (First two weeks in October) Presentation.
\end{itemize}

% Maybe useful things to cite:
% R. Milner "Well-typed programs can't go wrong" (1978)

% ~\cite{coquand1992pattern} % Dependent pattern matching is hard
% ~\cite{{quantitative-type-theory} % Blodwen implementation

% \bibliographystyle{plain}

\newpage
\printbibliography{}

\end{document}

% Not done: check possible half-time presentation
%   https://lists.chalmers.se/mailman/listinfo/proglog
%   https://lists.chalmers.se/mailman/listinfo/fp
